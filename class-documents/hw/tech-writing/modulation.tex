\documentclass[12pt]{article}
\usepackage[letterpaper, margin=1in]{geometry}

\usepackage{amsmath}
\usepackage{hyperref}
\usepackage{xspace}

\usepackage{graphicx}
\graphicspath{{images/}}

\usepackage{setspace}
\doublespacing

\usepackage[backend=biber,style=ieee,hyperref=true]{biblatex}
\addbibresource{references.bib}

\usepackage{fontspec}
\usepackage{kpfonts}
\fontspec{baskervaldx}
\setmainfont[Numbers=OldStyle]{baskervaldx}
\setmonofont[Scale=.9]{Courier New}

\newcommand{\amp}{\textit{\&}\xspace}

\newcommand{\ac}{\textsc{ac}\xspace}
\newcommand{\am}{\textsc{am}\xspace}
\newcommand{\dc}{\textsc{dc}\xspace}
\newcommand{\rf}{\textsc{am}\xspace}
\newcommand{\ssb}{\textsc{ssb}\xspace}

\DeclareMathOperator{\sign}{sign}

\title{\textsc{On SSB Modulation}}
\author{\vspace{-.8em} Catherine Van West \\
	\normalsize For \textsc{ece395} at The Cooper Union}
\date{Autumn, `23}

\begin{document}
\maketitle

\section*{Introduction}
Single-sideband modulation (hereafter \ssb modulation or just \ssb) is a
modulation scheme which decreases the bandwidth use of ordinary amplitude
modulation (\am) by a factor of two \autocite{ssb-thaddeus}. Ordinary \am
implemented using a single mixer produces two sidebands around the carrier
frequency (optionally with a strong component at the carrier itself). Since the
modulating signal is usually purely real, one of these sidebands is redundant
and may be eliminated without loss of information. Additionally, \ssb
modulation often suppresses the carrier, fully or partially, reducing the
amount of power needed to transmit the signal \autocite{weaver-rowell}.

This report examines the mathematics behind \am \amp \ssb, including a few
properties of the Hilbert transform. Basic familiarity with Euler's formula is
assumed. The treatment given here is specifically in light of the Hartley and
Weaver architectures \autocite{rf-microelectronics}, and briefly discusses the
advantages and disadvantages of each.

\section*{The Phasing Method}

\newcommand{\oin}{\omega_\text{in}}
\newcommand{\olo}{\omega_\text{lo}}

To illustrate the behavior of ordinary \am, consider a sinusoidal signal \(s(t)
= \cos \omega t\). Using Euler's formula, \(e^{jt} = \cos t + j \sin t\), we
may represent this as \(s(t) = \frac 1 2 (e^{j \omega t} + e^{- j \omega t})\).
Note that a single real sinusoid consists of two complex sinusoids at positive
and negative frequencies. If a baseband signal \(\cos \oin t\) is mixed with a
carrier signal \(\cos \olo t\), the result is
\begin{align*}
	\left(\cos \oin t\right) \left(\cos \olo t\right)
		&= \frac 1 2 \left(e^{j \oin t} + e^{-j \oin t}\right)
			\cdot \frac 1 2 \left(
				e^{j \olo t} + e^{-j \olo t}
			\right) \\
		&= \frac 1 4 \left(
			e^{j (\olo \pm \oin) t} + e^{-j (\olo \pm \oin) t}
		\right) \\
		&= \frac 1 2 \cos (\olo \pm \oin) t.
\end{align*}
(Note that \(\oin\) and \(\olo\) are intended to represent the radian
frequencies of the input signal and local oscillator signal, respectively.) The
modulated signal contains frequency components both above and below the
carrier, although these both have the same information content. The results of
modulation are illustrated in Figure \ref{fig:razavi-am}.

\begin{figure}[h]
	\centering
	\includegraphics[width=.9\textwidth]{razavi-am.png}
	\caption{Amplitude modulation of a baseband signal
	\autocite{rf-microelectronics}.}
	\label{fig:razavi-am}
\end{figure}

One means of suppressing one of the sidebands, known as the \emph{phasing
method}, is done using the Hilbert transform. The Hilbert transform is
equivalent to a filter with transfer function \(H(\omega) = -j \sign \omega\).
Applying this filter to a  sinusoid \(\cos \omega t = \frac 1 2 (e^{j \omega t}
+ e^{-j \omega t})\) yields \(\frac 1 2 (-j e^{j \omega t} + j e^{-j \omega t})
= \sin \omega t\) -- in other words, taking the Hilbert transform of a signal
shifts all real frequency components by \(\frac \pi 2\). For \ssb, the baseband
signal is modulated twice -- once in its original form, and once after passing
it through a Hilbert transformer \autocite{ssb-tretter}.

If this process is applied to the same sinusoidal signals as above, it yields
two sinusoidal signals, \(s(t) = \cos \oin t \) and its Hilbert transform
\(\hat s(t) = \sin \oin t\). These signals are then modulated with \(\cos \olo
t\) and \(\sin \olo t\), respectively, to produce in-phase and quadrature
components \(I(t)\) \amp \(Q(t)\) of the output:
\begin{align*}
	I(t) &= s(t) \cos \olo t \\
		&= \frac 1 4 \left(
			e^{j (\olo \pm \oin) t} + e^{-j (\olo \pm \oin) t}
		\right) \text{ by the above, and} \\
	Q(t) &= \hat s(t) \sin \olo t \\
		&= \frac 1 2 \left(-j e^{j \oin t} + j e^{-j \oin t}\right)
			\cdot \frac 1 2 \left(
				-j e^{j \olo t} + j e^{-j \olo t}\right
			) \\
		&= \frac 1 4 \left(
			-e^{\pm j (\oin + \olo) t} + e^{\pm j (\oin - \olo) t}
		\right) \text { after distributing.}
\end{align*}
The terms ``in-phase'' and ``quadrature,'' as used here, loosely refer to
similar signals out of phase by approximately \(\pi/2\). Note that, although
expressed in terms of complex sinusoids, both \(I(t)\) and \(Q(t)\) are
\emph{real} signals; the process above may be implemented in analog hardware.
Sideband selection is performed by either adding or subtracting \(I(t)\) and
\(Q(t)\) -- for example, adding yields
\begin{align*}
	I(t) + Q(t) &= \frac 1 4 \left(
		2 e^{j (\olo - \oin) t} + 2 e^{-j (\olo - \oin) t}
	\right) \\
	&= \cos (\olo - \oin) t \text{ after canceling \(\pm \sin\),}
\end{align*}
leaving only the lower sideband.

Demodulation functions as modulation in reverse -- the \rf signal is multiplied
by in-phase and quadrature components of a local oscillator, each resultant
signal is sent through an inverse Hilbert transformer, then the two components
are added back together to form the baseband signal. The Hartley architecture
is one example implementation, used since the early days of radio
\autocite{rf-microelectronics}.

\section*{The Weaver Architecture}

\newcommand{\oo}{\omega_o}
\newcommand{\oc}{\omega_c}

The phasing method allows \ssb generation with a single modulation step and
works with input signals of unknown bandwidth. The only difficulty with this
method is implementation of an accurate wideband Hilbert transformer. Using
\textsc{dsp} techniques on the original signal allows relatively easy
generation of its Hilbert transform \autocite{ssb-tretter}.

In analog circuits, however, the transform must be performed by a wideband
phase shift network, which makes rejection ratios above 40 dB quite difficult
to achieve. The Weaver architecture, reproduced in Figure
\ref{fig:weaver-from-paper}, offers an alternative method to generate \ssb
signals so long as the baseband signal is sufficiently bandlimited. The input
signal is modulated four times -- twice in both in-phase and quadrature -- and
lowpass filtered midway through. The main advantage of the Weaver method is
that the requirements on the filters are quite lenient, easing analog
implementation \autocite{weaver-himself}.

\begin{figure}[h]
	\centering
	\includegraphics[width=.9\textwidth]{weaver-from-paper.png}
	\caption{A Weaver modulator \autocite{weaver-himself}.}
	\label{fig:weaver-from-paper}
\end{figure}

If the same sinusoid, \(s(t) = \cos \oin t\), is modulated using the
above schema, the first in-phase and quadrature signals \(I_1(t)\) \amp
\(Q_1(t)\) (corresponding to \(e_{a1}\) \amp \(e_{b1}\) in Figure
\ref{fig:weaver-from-paper}) are
\begin{align*}
	I_1(t) &= \frac 1 4 \left(
			e^{j (\oin \pm \oo) t} + e^{-j (\oin \pm \oo) t}
		\right) \text{ (as before), and} \\
	Q_1(t) &= \frac 1 2 \left(e^{j \oin t} + e^{-j \oin t}\right)
			\cdot \frac 1 2 \left(
				-j e^{j \oo t} + j e^{-j \oo t}\right
			) \\
		&= \frac 1 2 \left(
			-j e^{j (\oo \pm \oin) t} + j e^{-j (\oo \pm \oin) t}
		\right).
\end{align*}
Lowpass filtering the above signals with a (perfect) cutoff at \(\oo\) yields
\begin{align*}
	I_f(t) &= \frac 1 4 \left(
			e^{j (\oo - \oin) t} + e^{-j (\oo - \oin) t}
		\right) \text{ and} \\
	Q_f(t) &= \frac 1 4 \left(
			-j e^{j (\oo - \oin) t} + j e^{-j (\oo - \oin) t}
		\right).
\end{align*}
Note that the lowpass filter's transition band can be as much as \(2 \oin\)
wide -- i.e., this band can be as wide as twice the lowest frequency component
of the input signal while still preserving functionality
\autocite{weaver-rowell}. After the second modulation stage and summation, we
have
\begin{align*}
	I_2(t) &= \frac 1 4 \left(
			e^{j (\oo - \oin) t} + e^{-j (\oo - \oin) t}
		\right) \cdot \frac 1 2 \left(
			e^{j \oc t} + e^{-j \oc t}
		\right) \\
		&= \frac 1 8 \left(
			e^{j (\oo - \oin \pm \oc) t}
			+ e^{-j (\oo - \oin \pm \oc) t}
		\right) \text{, and} \\
	Q_2(t) &= \frac 1 4 \left(
			-j e^{j (\oo - \oin) t} + j e^{-j (\oo - \oin) t}
		\right) \cdot \frac 1 2 \left(
			-j e^{j \oc t} + j e^{-j \oc t}
		\right) \\
		&= \frac 1 8 \left(
			e^{\pm j (\oo - \oin - \oc) t}
			- e^{\pm j (\oo - \oin + \oc) t}
		\right),
\end{align*}
so our transmission \(T(t)\) is (after rearranging)
\[
	T(t) = I_2(t) + Q_2(t) = \frac 1 4 e^{\pm j (\oc - \oo + \oin) t}
		= \frac 1 2 \cos\, (\oc - \oo + \oin) t,
\]
an upper \ssb signal centered at \(\oc - \oo\).

\section*{Practicality}
D. K. Weaver gives a practical implementation example (reproduced in Figure
\ref{fig:weaver-impl}), which, though dated, serves to demonstrate the
simplicity of the Weaver method. The circuit comprises sixteen diodes and eight
transformers, four of which are air-core high-frequency types. Given its
relative simplicity, the circuit is an attractive candidate for a simple analog
\ssb radio -- which was its original intent. Both of the above methods of
modulation are studied and used in both transmitters and receivers
\autocite{ssb-thaddeus}, \autocite{rf-microelectronics}.

\pagebreak

\begin{figure}[h]
	\centering
	\includegraphics[width=.9\textwidth]{weaver-impl.png}
	\caption{An implementation of a Weaver modulator
	\autocite{weaver-himself}.}
	\label{fig:weaver-impl}
\end{figure}

\printbibliography

\end{document}

\documentclass[12pt]{article}
\usepackage[letterpaper, margin=1in]{geometry}

\usepackage{amsmath}
\usepackage{hyperref}
\usepackage{xspace}

\usepackage{setspace}
\doublespacing

\usepackage[backend=biber,style=ieee,hyperref=true]{biblatex}
\addbibresource{references.bib}

\usepackage{fontspec}
\usepackage{kpfonts}
\fontspec{baskervaldx}
\setmainfont[Numbers=OldStyle]{baskervaldx}
\setmonofont[Scale=.9]{Courier New}

\newcommand{\amp}{\textit{\&}\xspace}

\newcommand{\ac}{\textsc{ac}\xspace}
\newcommand{\am}{\textsc{am}\xspace}
\newcommand{\dc}{\textsc{dc}\xspace}
\newcommand{\fm}{\textsc{fm}\xspace}
\newcommand{\ssb}{\textsc{ssb}\xspace}

\title{\textsc{On SSB Modulation}}
\author{Catherine Van West}
\date{Autumn, `23}

\begin{document}
\maketitle

\section{Introduction}
Single-sideband modulation (hereafter \ssb modulation or just \ssb) is a
modulation scheme which decreases the bandwidth use of ordinary amplitude
modulation (\am) by a factor of two \autocite{ssb-thaddeus}. Ordinary \am
implemented using a single mixer produces two sidebands around the carrier
frequency (optionally with a strong component at the carrier itself). Since the
modulating signal is usually purely real, one of these sidebands is redundant
and may be eliminated without loss of information. Additionally, \ssb
modulation often suppresses the carrier, fully or partially, reducing the
amount of power needed to transmit the signal \autocite{weaver-rowell}.

This report examines the mathematics behind \am \amp \ssb, including basic
properties of the Hilbert transform. It also gives an overview of a few
potential implementations, including both Hartley and Weaver modulators, and
discusses the advantages and shortcomings of each.

\section{Amplitude Modulation}


\printbibliography

\end{document}

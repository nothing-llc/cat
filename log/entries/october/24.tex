\subsection*{engineering}
returned to the problem of the hilbert transformer and kept simulating \amp
programming. razavi, once again, has the answer! if we upconvert the original
signal, then put it through an \emph{approximate} \(\pi/2\) phase shift, we'll
get out what is very nearly a hilbert transform of the original! this \emph{is}
the hartley architecture -- thank goodness he actually stated it!

\ldots assuming i correctly understand how it works, that is.

after some investigation\ldots\ hartley may not be the way to go. razavi
mentions the weaver architecture, which requires four (!) mixers but has better
phase and amplitude behavior, so we may try that one instead. the basic idea is
to modulate the audio signal up a bit, filter it, then modulate it all the way
to the \rf carrier.

initial implementation of a weaver modulator (same code file as yesterday) went
well, and required very little filtering to do so -- just a fourth-order
chebyshev ii filter was sufficient in my initial test case.

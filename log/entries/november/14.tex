\subsection*{engineering}
working from home today, as i'm likely contagious. decided to play with
obtaining a quadrature signal from the \pll without using a coupled
oscillator\ldots\ and i found a way (see
\localfile{../work/pll-shifter/pll-shifter.asc}). i basically made a
differentiator to do it, which would be straightforward except that we don't
have op-amps which run at 14 MHz. so i took a common-base amplifier, fed a
capacitively-coupled input in, and took the quadrature signal from the output
(the capacitor produces a 90\textdegree\ phase shift between voltage and
current, and the common-base amp, due to its low input impedance, basically
converts that into a voltage). to improve the quality of the shift (up to
87\textdegree\ if we're feeling expensive or 85\textdegree\ if we want to save
power), i added some negative feedback which decreased the input impedance even
further. i shall remember this technique for making low input impedance
amplifiers at high frequencies; it was quite handy!
